\documentclass[french, a4paper, 12pt]{article}

\newcommand{\docname}{\sc Les révolutionnaires boisais}
\newcommand{\authorname}{}
\newcommand{\dmy}{}
\newcommand{\years}{2022}
\newcommand{\org}{Odyssée}

\input{~/latex/article.tex}

\begin{document} \maketitle \vspace{3pt} \hrule \vspace{3pt}

\section{Arrivée à Boisia}
L'après-midi pluvieuse touchait à sa fin, et avec elle le long voyage de Narcisse et Morgan. Le ciel lourd et gris les accompagnait depuis plusieurs heures et voir au loin la faible lueur des torches de la mesnie de Frolorn leur procurait un certain sentiment de gaité malgré la perspective peu réjouissante de durs évènements à venir.\\

Percée dans un épais mur de pierres inégales, la lourde porte en bois et fer forgé de la mesnie sembla émerger de brume, encadrée par deux torches grésillantes. Une ombre se détacha de la paroi et s'avança vers nos voyageurs. À la lueur tremblotante des torches, Narcisse devina un homme lourdement armé.\\
--- Narcisse~? demanda l'homme\\
--- C'est moi. répondit l'intéressé\\
--- Ne restons pas là, vous allez attraper la mort.\\
Narcisse et Morgan suivirent l'homme dans le bâtiment. La salle principale, était froide malgré la cheminée, mais elle conservait néanmoins un certain charme, le plafond était plutôt haut, ponctués par de grosses poutres apparentes. Le sol était recouvert d'un plancher rustique, et quelques hommes en armes étaient assis ici et là. Certains se restauraient, d'autres jouaient aux cartes ou aux dés leurs maigres soldes. Le garde qui les avait accompagnés jusqu'ici se tourna vers eux~:\\
--- Frolorn ne devrait pas tarder à venir. Il vous précisera ce que vous devez savoir.\\
Et, sans attendre de réponse, la sentinelle se remit en marche en direction de la porte.\\

Frolorn ne tarda pas à arriver. Saluant les nouveaux-venus.\\
--- Bienvenue dans ma mesnie. Les temps sont durs pour nous et je ne veux pas que vous ayez de contacts avec les autres pour l'instant. Vous n'aurez donc pour seul compagnon que Pyracmon, au moins dans un premier temps.\\
Le Troll jusqu'alors sagement assis dans un angle se leva, manquant de renverser la table devant laquelle il était. D'une démarche rendue maladroite par la pièce trop étroite pour lui Pyracmon s'avança vers Narcisse et Morgan.
--- Bonjour à vous, mon jeune et brave compagnon~! Le salua Narcisse puis il continua~: C'est une longue route que nous allons être amenés à parcourir côte-à-côte, et je me réjouis de la parcourir à vos côtés. \\
--- bonjour. Vous nouvel amis~?\\
Narcisse eut un mouvement instinctif de recul tandis que Morgan secouait la tête d'un coup sec et martial en guise de salutation, sa main ne quittant pas pour autant le pommeau de sa dague. N'obtenant pas de réponse, Pyracmon entreprit de nettoyer sa lunette.\\
--- Oui~! Nous sommes des amis~! répondit Morgan se disant qu'une absence de réponse pourrait avoir des conséquences saignantes.\\
--- Qui doit être tapé~? Demanda le Troll, impatient d'en découdre.\\
--- Comme vous voudrez, mon jeune… enfin… cher ami… euh… monsieur~! tenta de répondre Narcisse \\

Frolorn interrompit les présentations et invita les nouvelles recrues à aller se restaurer pendant qu'il leur expliquerait les détails de leur première mission. Narcisse se pencha vers Morgan~:\\
--- Il faudra que vous m'indiquiez comment procéder~: je n'ai pas l'habitude d'avoir un supérieur hiérarchique\\
--- J'ai toujours eu un problème avec l'autorité, mais je crois que tant que vous le laissez en vie et entier, il n'y aura pas de problème… répondit Morgan en haussant les épaules.\\
--- Ah, parfait, j'essaierai.\\
Narcisse et Morgan se rendirent compte que Frolorn les regardait et arrêtèrent de discuter à voix basse.\\
--- Bien. Vous allez devoir intercepter un messager à cheval. Il arrive de Sérésia… commença le maître des lieux\\
--- Je pourrais manger le cheval~? demanda Pyracmon le plus sérieusement du monde.\\
--- Euh… non. Nous avons besoin de chevaux pour mener à bien notre révolution.\\
Narcisse se tourna vers Frolorn~:\\
--- Et comment identifiera-t-on le messager~?
--- Il portera les armoiries de Sérésia, vous devriez pouvoir le reconnaître, elles sont assez simples. Je vous ferais porter une réplique d'un uniforme dans la soirée.\\
--- Les armoiries… il les portera comment~? Parce que si c'est sur une chevalière, je ne suis pas certain que nous ayons la vue assez perçante pour… s'inquiéta Narcisse\\
--- Moi je vois bien. Très bien avec mon gros œil~! une fois j'ai réussi à voir une fourmi~! le coupa Pyracmon.\\
--- Non, il s'agit d'une tunique explicita Frolorn, vous ne devriez pas avoir de problème.\\

Le meneur de la révolution guida Narcisse, Pyracmon et Morgan jusqu'à une table de bois portant les stigmates du temps et sur laquelle s'étalait pains, vins, fromages et autres sortes de victuailles.\\
--- Prenez des forces, vous en aurez besoin demain.\\
--- Ah, merci mon jeune… euh… monsieur, pour ce repas que vous nous offrez avec bienveillance. Le remercia Narcisse en train de lorgner sur la table avec un appétit non dissimulé. Sentant qu'il fallait une confirmation de sa part pour l'interception du lendemain, l'incarnation divine se retourna brusquement vers Frolorn~: Soyez assuré que nous capturerons sans faute votre messager~!\\

Les trois compagnons avaient déjà bien attaqué les mets lorsque Frolorn revint, un bout d'ettofe bleue à la main~:\\
--- Ce sont leurs armes. Les nôtres sont rouges, ne réfléchissez donc pas trop, si c'est bleu, tirez.\\
Pyracmon qui a compris qu'il ne devait pas réfléchir, répliqua du tact-au-tac~:\\
--- Toi petit homme qui parle bien, je te suis pour ces fromages~! \\
Morgan, jetant un regard inquiet à Pyracmon et se souvenant de son daltonisme prononcé, fit part de ses doutes à Frolorn~:\\
--- Et, dans l'absurde hypothèse où on confondrait les deux couleurs~?\\
--- Si vous deviez confondre les deux couleurs, alors vous aurez tué un homme innocent, voire un des nôtres. Vous seriez alors pendus haut et court, comme traitres. Répondit Frolorn.
--- Je dois taper quand c'est rouge c'est ça~? demanda Pyracmon pour confirmation. \\
--- Oui répondit Morgan sans y réfléchir. \\

Narcisse prit soudain conscience qu'il ne s'est pas présenté à son hôte~:\\
--- Mon nom est Narcisse le G… Tout-court. Narcisse Tout-court\\
Pyracmon le toisa depuis ses trois mètres de haut~:\\
--- Je sais que tu es tout court.\\
Devant l'échange de répliques, Morgan s'empressa de fourrer un bout de pain dans sa bouche pour étouffer le rire naissant.\\

Narcisse se tourna vers Morgan, et sur un ton de reproche~:\\
--- Quelle idée vous avez eue, aussi, de naitre daltonien~?\\
--- J'ai pas choisi… se défendit Morgan, noyant sa frustration dans une orgie de pain.\\
--- C'est où Altonien comme lieu~? demanda Pyracmon\\
--- Vous auriez du… le sermonna Narcisse, puis, se tournant vers Pyracmon, C'est une contrée lointaine, où les gens inversent toutes les couleurs. Le jaune devient vert, le vert devient orange…\\
--- Ils sont fous~! lança Pyracmon.\\
--- C'est terrible… renchérit Morgan\\
--- L'orange devient bleu canari… poursuivit Narcisse\\
--- Ça tombe bien j'aime pas les oranges. conclut le Troll\\
--- Et le bleu canari devient blanc ver de terre termina le mage\\
--- AH~! cria Pyracmon, horrifié à l'idée que l'on puisse manquer de respect à de la nourriture.\\

Frolorn entra dans la salle principale et vous demanda de la suivre~:\\
--- Je vais vous montrer vos couches.\\
La petite équipe hétéroclite suivit Frolorn dans le dédale de pièces et de couloirs de sa mesnie jusqu'à une petite pièce mal chauffée. La sensation de froid était d'autant plus forte que la pièce était éclairée par une unique chandelle de suif qui dégageait une forte odeur de graisse brûlée. Quelques couches de pailles était jetée à même le sol. Une chaise en bois et un seau d'eau pour les ablutions complétaient le pauvre mobilier. Frolorn continua à prodiguer ses conseils~:\\
--- Dormez, demain sera une journée bien remplie.\\
Mais déjà Pyracmon, ivre, s'était effondré au sol et ronflait à en faire trembler les murs.\\

\section{Interception}
Le jour pointait à peine, la petite chandelle de suif s'était éteinte il y a longtemps. Frolorn entra avec un grand fracas dans la pièce, vous hurlant de vous lever. Morgan, le premier à ouvrir un œil se leva péniblement, suivi par Narcisse. Tous deux regardèrent Pyracmon, beuglant dans son sommeil tout en essayant de manger un pied de chaise~:\\
--- TOUCHE PAS MON FROMAGE~!\\
Morgan et Narcisse échangèrent un regard consterné.\\
--- Bon bon bon… lança Morgan avant de lancer le seau d'eau froide sur Pyracmon lequel se réveilla en sursaut\\
--- Pourquoi tu m'as fait ça~?\\
--- Tu ne voulais pas te réveiller, alors moi, grande âme, je t'aide. répondit Morgan avec un sourire ambigu.\\
Mécontent, Pyracmon se lève en boudant~:\\
--- Grumph… Ça me rappelle mon dernier bain…
--- Je ne veux pas de détail à ce sujet~! l'interrompit Narcisse se bouchant le nez de son manteau pour tenter de contrer l'odeur pestilentielle qui se répandait dans la pièce exigüe.\\
--- Quels détails~? Il n'a jamais pris de bain. lança Morgan, un sourire aux lèvres.\\
--- Si~! Je me lave d'abord~!\\
Sentant que la situation lui échappe un peu, Morgan, dans un élan de génie stratégique, plaça Narcisse entre lui et Pyracmon. Ce dernier poursuit sur les modalités du bain~:\\
--- Quand la lune cache le soleil je peux me laver sans être gêné par le soleil.\\
--- Il faut que j'apprenne à provoquer des éclipses. se dit Narcisse pour lui-même\\
--- Oui, ça me semble pertinent…\\

Pyracmon, irrité ne rien comprendre lança un~\og Allons-y \fg~franc et massif, tout en titubant vers la porte. L'initiative fut approuvée par Morgan. Narcisse ayant mis la main sur une carte, la petite troupe sortie dans les rues de Solle. Le ciel était clair et le jour serait bientôt complètement levé. La mesnie faisait face aux docks et à la mer. D'après Narcisse, le messager allait arriver par la route du sud, il fallait donc rejoindre l'axe principal pour ensuite retrouver l'entrée de la bourgade. Le soleil terminait de chasser les dernières ombres de la nuit lorsque les trois compagnons arrivèrent à l'entrée de la ville. La rue principale est chaussée de pavés inégaux. Malgré l'heure matinale, une foule de badaud déambule dans les rues, marchands, visiteurs, artisans… charrettes et cavaliers transitent, se croisent, s'insultent pour une priorité. En cherchant dans la foule un cavalier à tunique bleue, Pyracmon demande à Narcisse~:\\
--- Tu as le message~? déjà~?\\
Morgan regarde Narcisse et sa carte avant de répondre~:
--- Hum, oui, mais on aimerait tuer le messager en bleu. Parce qu'on aime pas le bleu.\\
--- C'est le rouge non~?\\
--- C'est pareil. rétorqua Morgan\\
--- Mon jeune ami, ne commencez pas à troubler ce garçon~! Il a déjà suffisamment de mal comme ça.\\
Morgan regarde ses pieds, penaud,\\
--- Moui…\\

Plusieurs cavaliers passent jusqu'à ce que Pyracmon s'exclame~:\\
--- CAVALIER~!\\
Interloquée, l'intéressée se retourna sur sa selle, dévoilant des traits féminins.\\
--- Excusez-moi gente dame, de quelle couleur êtes-vous~? demanda Narcisse avec une incroyable subtilité.
Étonnée, la cavalière tourna la bride et s'éloigna sans demander son reste.\\
--- Je crois qu'elle était en marron de toute façon… avança Narcisse pour soulager sa conscience.\\
Mais déjà un autre cavalier attirait l'œil du mage.\\
--- Là du r… bleu~!\\
--- Où ça~?\\
--- Là-bas répondit Narcisse en désignant un cavalier qui remontait à vive allure la route du sud en direction du nord. La distance permettait déjà de deviner le bleu de sa tunique.\\
--- Il faut être sûr qu'il s'agisse bien d'un sérésien et pas d'une bête tunique bleue. tempéra Morgan.\\
Pyracmon ne prend pas la peine d'essayer de voir et boude dans son coin. Narcisse d'approche de lui~:\\
--- Allons, mon garçon, pourquoi faites-vous cette tête~?\\
--- J'ai pas pu taper la première…\\
Le cavalier continue de se rapprocher à bride abattue jusqu'à la confirmation de Morgan~:\\
--- C'est la bonne tunique~! Pyracmon~? Il est pour toi~!\\
--- Les gens ne risquent pas d'intervenir~? s'interrogea Narcisse\\
Levant brutalement la tête, Pyracmon vit rouge à la vue de la tunique bleue et se rua sur le cavalier en hurlant. Le cavalier surpris tenta de faire demi-tour en voyant le Troll se jeter sur lui, mais trop tard~: Pyramon l'avait déjà soulevé au-dessus de son cheval et le portait délicatement par les aisselles, savourant son moment. Pyracmon retourna le message terrifié et lui fracassa la tête contre les pavés dans une explosion de sang. Tout content de sa prouesse, il ramena le message imbibé de sang à Narcisse qui tentait de rassurer les manants choqués. Machinalement, Pyracmon nettoya sa hache sur la cape du messager.\\
--- Mais… elle est propre ta hache~? nota Morgan\\
--- Bah non. répondit Pyracmon avant de noter pour lui que les tout-courts sont bêtes.\\
Narcisse mit fin à la discussion en pointant du doigt un groupe d'une dizaine d'hommes armés.\\
--- Ce n'est pas la maréchaussée qui vient vers nous~?\\
--- C'est très possible, allons-nous en. proposa Morgan en prenant le cheval par la longe.\\

L'équipée se faufila dans une ruelle perpendiculaire dans l'espoir de semer la maréchaussée. Après quelques minutes de course dans le dédale de rues de la Vielle Ville, Narcisse s'arrête, adossé au mur d'un bâtiment effondré où la lumière du soleil est un peu plus présente.\\
--- Essayons de lire ce message…\\
Pyracmon arrache le message des mains de Narcisse.\\
--- Je vais le lire.\\
--- Tu sais lire~? s'étonna Narcisse, en état de choc\\
--- Euh… Ça parle de fromage… et euh… d'acheter de la bière… tenta de déchiffrer Pyracmon\\
--- Bon, re-donnez-moi ça jeune homme.\\
Narcisse tendit la main et récupéra le message.\\
--- Bonne nouvelle c'est du Luarian, mauvaise nouvelle une partie du message est recouverte par le sang. Il est effectivement question de fromage… Ah non~! Rassemblement armé~! C'est presque le même mot…\\
Narcisse allait continuer lorsque Pyracmon signala avoir vu une ombre passer sur le mur.\\
--- Ne restons pas là. souffla Morgan\\
--- Il ne faut pas rentrer directement à la mesnie. suggéra Narcisse\\
--- Bonne idée~: allons à l'auberge~! lança un Pyracmon joyeux à l'idée d'avoir du fromage.\\
Une ombre passa sur le mur de la maison en ruine que vous venez de quitter. Sans hésiter, Pyracmon dégaina sa hache, le fer affuté siffla en fendant l'air et vint se ficher en plein dans la tête… de l'ombre du Troll. Non content de s'arrêter en si bon chemin dans son combat, Pyracmon dégage sa hache à deux mains du pavé dans lequel la lame s'était enfoncée et entreprend de défoncer le pavage sur quelques mètres.\\
--- VICTOIRE~! hurla Pyracmon lorsqu'il trouva une pièce. Jugeant que son ombre était morte, il rangea sa hache et rejoignit Narcisse et Morgan qui le regardait sans comprendre.\\

La troupe déambulait dans la vieille ville à la recherche d'une auberge pour se restaurer sous un soleil maintenant haut dans le ciel. Le cheval était toujours tenu en longe par Morgan et l'impression d'être suivi ne les quittait pas et il fallait un endroit fréquenté. Une gargote à l'aspect lugubre répondait à ce critère. Morgan attacha le cheval à un anneau fixé dans le mur et nos trois héros poussèrent la mince porte vermoulue et déformée par l'humidité pour entrer dans une pièce basse de plafond, au sol de terre battue et imprégnée d'une odeur de charbon. De longues tables sont alignées et la salle est bondée. L'équipe se fraye avec difficulté un chemin dans la foule et l'épais silence qui accompagne l'entrée du Troll, jusqu'au comptoir où il reste un peu de place. Certains clients se raidissent à la vue de Pyracmon, manifestement, ce dernier a laissé quelques souvenirs lors de sa dernière visite. Le Troll, pas le moins gêné du monde, hèle une barmaid~:\\
--- À boire, et du fromage~! Vous autres faites moins de bruits~! Mon ami doit déchiffrer un message~!\\
Narcisse, dont le visage reste d'une impassibilité incroyable déclame~:\\
--- Bonjour braves gens~! Ne nous prêtez pas d'attention, mon ami est soul~!\\
Peu à peu, le brouhaha envahit de nouveau l'établissement tandis que pains, fromages et choppes de bière passaient sur leur bout de table.
--- Tu parles bien toi, il faudra m'apprendre à faire ça un jour. dit Pyracmon\\
--- Je le ferai mon garçon, je le ferai. promis Narcisse.\\
Narcisse profite du départ d'un homme pour s'asseoir et entreprendre de lire le message plus en profondeur. De son côté, Morgan, intrigué par le départ de l'homme tente de voir où il va, mais la foule est si dense qu'il ne parvient pas à avancer et fini par rejoindre ses coéquipiers. Pendant ce temps Narcisse a avancé dans sa traduction.\
--- Le sens reste flou, Sérésia voudrait envoyer un détachement armé… Dans son pli, le gourvernement sérésien fait part d'une satisfaction devant la tournure des évènements. Un certain plan en plusieurs étapes est également évoqué. Sérésia va peut-être implanter une garnison permanente à Boisoi, mais… Où est passé notre Troll~?\\
Un lourd fracas ébranle la petite taverne suivi d'un hurlement de victoire.\\
--- Je crois qu'on l'a retrouvé. note Morgan.\\
--- Je crains le pire…\\
--- Il a l'air d'aller bien… note que je n'en dirais pas autant de l'auberge.\\
Derrière Narcisse, Pyracmon s'est levé, triomphal, une table brisée en deux jonche le sol et les hurlements d'un homme dont le bras forme un angle anormal ne parviennent pas à couvrir les cris de la foule en délire qui acclame le Troll.\\
--- J'ai gagné au bras de fer. explique ce dernier. Vous avez réussi~?\\
Narcisse dépité, se retourne finalement.\\
--- Mon garçon, vous êtes décidément une source d'étonnement toujours renouvelé… Sinon, oui, j'ai partiellement réussi.\\
--- Les nouvelles ne sont pas bonnes, il faut que l'on retourne à la mesnie. ajouta Morgan.\\

\section{Maréchaussée et garnison}
Durant le trajet qui séparait l'auberge de la mesnie, Narcisse et Morgan s'employèrent à expliquer à Pyracmon les enjeux de Sérésia ainsi que le contenu du message. Arrivé à destination, Frolorn attendait, adossé au mur du bâtiment principal.\\
--- Messire~! le salua Narcisse puis, à l'oreille de Morgan~: Vous avez vu, je ne l'ai pas appelé~\og mon jeune ami \fg~!\\
--- Vous êtes en progrès~!\\
--- Merci, puis à Frolorn, Notre mission est une réussite totale~!\\
--- Vous me raconterez à l'intérieur comment un massacre à la hache peut-être qualifiée de réussite totale. répondit Frolorn, sceptique.\\
Une fois que Narcisse eut terminé le récit de leur mission et remit le message à Frolorn, ce dernier s'adoucit et reconnut que c'était un succès.\\
--- Prenez du repos, cette maison est maintenant la vôtre, mangez, dormez. Il faut que j'aille m'entretenir avec mes conseillers sur la marche à suivre. Ce projet de garnison permanente basée à Courte ne me plaît pas.\\
Les trois amis s'allongèrent en pestant contre le manque de confort.

Forlorn revient quelques instants plus tard~:\\
--- Nous allons frapper un grand coup~!\\
--- Déjà fait. répondit Pyracmon sans ouvrir un œil.\\
--- Lorsque la garnison sérésienne se rendra à Courte, elle fera passera par Solle qui est la dernière bourgade, il leur restera encore une cinquantaine de kilomètres à parcourir. Nous les intercepteront à ce moment-là.\\
--- À trois contre une garnison, ça risque d'être compliqué. nota Morgan\\
--- Nous allons réunir un grand nombre de partisans et nous auront l'avantage de la surprise. D'autant plus qu'une garnison n'est pas constituée que de soldats, il faut également du personnel administratif, des cantiniers et autres personnels. Nous saurons les accueillir.\\
Pyracmon qui n'a pas tout suivi~:\\
--- Ils auront du canard~?\\
Frolorn ignora l'interruption et poursuivit~:\\
--- Moi et le gros de mes hommes allons nous charger de la garnison à proprement parler. Vous, vous empêcherez la maréchaussée sollaise d'intervenir pour protéger la garnison.\\
--- Ça me semble clair dit Morgan.\\
--- Il y a beaucoup d'hommes dans la maréchaussée de Solle~? s'inquièta Narcisse\\
--- Non, une dizaine maximum. De plus la majorité est boisaise et, bien qu'attentiste, elle n'opposera pas de résistance. Il faudra vous focaliser sur les quelques soldats sérésiens présents. Mais attention, ce sont des combattants d'élite~!\\
--- Comme moi~? demanda Pyracmon en bombant le torse.\\
--- Non pas comme toi… répondit Frolorn en levant les yeux au ciel. Ils ont l'art de se battre\\
Morgan prit la défense de Pyracmon~:\\
--- Pyracmon aussi a lard de se battre.\\
--- Morgan, vous êtes un cochon~! s'exclama Narcisse\\
--- Grouïïïk~?\\
--- Bon, la maréchaussée a ses quartiers pas très loin du port, et le message est trop effacé pour que l'on sache quand la garnison arrivera. En attendant reposez-vous, nous ne nous reverrons pas avant la bataille, c'est plus prudent. les coupa Frolorn, agacé.\\
--- Et si on allait manger~? proposa Pyracmon dont les bruits de cochons avait réveillé l'appétit.\\

La troupe réfléchissait à un plan pour mieux attaquer la maréchaussée. Narcisse défendait une infiltration pour savoir qui est qui, Morgan proposait d'aller acheter des armes et Pyracmon ne comprenait pas pourquoi ils n'étaient pas encore en route. Après un dialogue compliqué dans lequel Pyracmon manqua de détruire le toit de la mesnie, il finit par convenir qu'il fallait d'abord acheter des armes. La troupe quitta donc la mesnie, suivant le Troll qui amena tous le monde chez Calashite, un forgeron réputé dont la forge est située au nord de Solle, dans les rues de la Vieille ville.\\
--- Ferdinant~! s'écria Pyracmon en défonçant la porte de la forge\\
--- Ferdinant c'est le boulanger, Pyracmon… Calashite s'aperçut alors que le Troll n'était pas venu seul~: Oh~! bien le bonjour~!\\
--- Bonjour. répondit Morgan avec un hochement de la tête\\
--- Enchanté~! C'est un plaisir que de vous rencontrer~! Nous avons beaucoup entendu parler de vous~! puis plus bas, à Morgan~: la flatterie mon ami, notez bien, la flatterie \\
--- L'hypocrisie, je ne m'y ferais jamais…\\
--- Je l'ai apprise au cours d'un voyage chez les sorciers nains de Torkia, qui pratiquent une forme d'hypocrisie magique qui leur ouvre toutes les portes de la Cour et de l'administration.\\
La forge est spacieuse, l'air est sec et chaud, le ronflement des feux vous parvient depuis la boutique, et les cheminées crachent leurs fumées dans l'air frais de la mi-après-midi. Tout autour de vous des armes et armures d'une qualité… reprochable sont exposées. Calashite emmèna les trois compagnons dans une petite pièce attenante où la qualité était bien meilleure.\\
--- Ah oui, voila qui est mieux, il nous faudrait des armes de jets. dit Narcisse.\\
--- Oui c'est une bonne idée. renchérit Morgan.\\
--- Oui, bien sûr, s'inclina Calashite.\\
--- Ah, j'oubliais… Qui sait manier une arme de jet~? s'interrogea Narcisse\\
--- C'est pas plus dur qu'une épée… De toute façon tu as déjà prouvé ton inefficacité totale au combat.\\
Piqué dans son orgeuil, Narcisse abattit sans crier garre sa lourde sacoche de grimoires sur le crâne de Morgan qui couina sous le choc.

Après quelques autres péripéties et erreurs de calculs de la part de Pyracmon, tous trois quittèrent la forge laissant Calashite un peu plus riche. L'après-midi est bien avancée lorsqu'ils se dirigèrent vers l'ouest pour honorer le funeste rendez-vous avec la maréchaussée. Mais au bout de quelque pas sur les rues pavées de la Vieille ville, Morgan s'arrête et demande~:\\
--- Mais… il est où le Troll~?\\
Un cri se fit entendre dans une ruelle voisine. Lorsque Narcisse et Morgan arrivent sur les lieux, une passante manifestement évanouie est allongée par terre, Pyracmon qui allait tenter de la réveiller, se retourne au bruit des pas.\\
--- Ah~! J'allais lui demander le chemin, et elle est tombée.\\
Pyracmon entreprend de lui tapoter la joue et ce faisant lui arrache la tête qui roule jusque sous les pieds d'un garde de la maréchaussée en pleine ronde.\\
--- Hé~! Qu'est-ce qu'il se passe là-bas~?\\
--- Euh rien~? répondit Morgan en faisant semblant de ne pas voir la tête\\

Narcisse, Pyracmon et Morgan commençaient à tourner les talons lorsque le garde les arrêta pour de bon.\\
--- Vous~! Arrêtez~!\\
--- Oui~? demanda Morgan d'un ton innocent\\
--- Que se passe-t-il mon brave~? interrogea Narcisse\\
Pyracmon tente d'intimider le garde nullement impressionné par la masse du Troll.\\
--- C'est notre Troll de compagnie qui a causé un peu bruit~? continua Narcisse\\
--- Du bruit, si seulement~! répliqua le garde en montrant le corps tordu et déchiré.\\
--- Elle s'est tuée toute seule~! argumenta Pyracmon, elle est tombée, j'ai voulu l'aider et sa tête s'est décollée.\\ 
--- Je vais vous demander de me suivre jusqu'à la caserne où vous serez emprisonné en attendant votre jugement.\\

Après quelques minutes de marches sous un ciel un peu voilé, la petite troupe arrive en vue de la caserne. Le bâtiment est large et imposant. Le rez-de-chaussée est fait de pierres, remplacée par des colombages et du torchis à base de paille dès le premier étage. Quelques petites briques rouge sombre affleurent du mur et viennent consolider l'ensemble. Une courte cheminée émerge du toit de chaume pour cracher sa fumée noire et grasse dans l'air sec et frais.\\
--- Nous y voila. nota Morgan\\
--- Un endroit ma foi fort accueillant. souligna Narcisse.\\
Le garde qui tient Pyracmon se retourne vers vous, un air contrit,\\
--- Je suis désolé, vous allez devoir rester dans la salle commune.
La pièce en question est plutôt spacieuse, mais spartiate. Le sol est composé de briques en quinconces partiellement recouvertes de terre. Une table encadrée de deux bancs occupe un angle, un grand coffre un autre. Des râteliers à demi-plein portent des armes~: épées, masses… Le sol est percé d'une trappe qui descend vers les geôles et un petit escalier monte au premier étage où se trouvent les dortoirs. Quelques gens d'armes présents se retournent vers vous. Narcisse, lequel a entre-temps entrepris d'attendrir le garde se lamente~:\\
--- Nous prendrons notre mal en patience en gémissant au souvenir de notre regretté camarade. Et quelquefois, quand la lune sera dans son premier quartier et que les vaches meugleront sur le bord du chemin, nous nous arrêterons, et nous nous souviendrons de celui qui fut notre ami.\\
--- Je promets de tout faire, il aura la vie sauve. répondit le garde en se retournant une dernière fois, la larme à l'œil.\\
--- Merci, merci mon brave, vous êtes un homme bon, je suis sûr que vous êtes boisais de naissance~!\\
Le garde bomba le torse et haussa la voix~:\\
--- Oui monsieur~! Boisais~!\\
À cette phrase, les quatre autres soldats présents se retournèrent et jetèrent un regard sombre sur le garde qui garda la tête haute. C'est le moment que choisit Pyracmon pour se lancer dans un discours élogieux et plein de vigueur pour soutenir le garde.\\
--- Fie… Jeee… Picougnac mon ami~!\\
Devant la tête ahurie de ce dernier, Narcisse lui demanda~:\\
--- Dites-moi, vos quatre amis, là-bas, ce sont des sérésiens, n'est-ce pas~?\\
--- Ah ça c'est bon le sarrasin~! dit Pyracmon en bavant un peu.\\
--- Oui… chuchota le garde, mais ils ne sont pas trop appréciés par ici…\\
--- Vous voulez dire que… vous voudriez vous en débarrasser~? proposa Morgan que la perspective d'un combat excitait.\\
Le garde jeta un regard choqué à Morgan qui haussa les épaules.\\
--- Ce genre de… méthode ne me plaît pas. Et puis ce sont de braves gars, dans leur genre…\\
--- Mon ami a toujours eu des difficultés à s'exprimer, pardonnez-lui. expliqua Narcisse avec un sourire de dentiste et en essuyant un regard noir de la part de Morgan.\\
--- Je comprends tout à fait.\\
--- Et… il y a d'autres sérésiens ici~?\\
--- Hum, pas que je sache. Solle est une petite bourgade, la maréchaussée n'a pas besoin de compter beaucoup d'homme.\\
--- Nous n'avons pas besoin de tous ces étrangers chez nous, je pense. poursuivit Narcisse à voix basse.\\
--- Je ne pense pas non plus, mais bon, c'est comme ça, que voulez-vous…\\
--- Oui, nous ne pouvons pas y faire grand-chose…\\
Le garde sembla redécouvrir avec surprise l'existence de Pyracmon et mit fin à la conversation. Le Troll allait suivre le garde dans la trappe quand Narcisse s'approcha du Troll et lui dit à voix basse~:\\
--- Quand vous entendez un appel à l'aide, vous êtes prêt à l'apporter par tous les moyens~?\\
--- Bien sûr~!\\
--- Alors soyez attentif Pyracmon, un appel peut toujours retentir… déclama Narcisse le doigt levé.\\

À peine le garde boisais était-il remonté que Narcisse se jeta vers lui~:
--- Dites-moi mon brave, quand finit votre service~?\\
--- En temps normal vers 18 heures, mais aujourd'hui, un détachement armé va traverser la ville en direction de Courtes. Je ne sais pas vers quelle heure tout cela sera fini…\\
--- Un détachement armé~? Mais quel évènement extraordinaire ~ Et vous allez veiller à ce que son passage se déroule sans encombre, c'est ça~?\\
--- Ehum, oui c'est cela.\\
--- Et vos collègues, là, ils vont faire la même chose~?\\
--- Plus personne pour surveiller notre ami alors~? interrogea Morgan, déçu par ce service de piètre qualité.\\
La voix de Pyracmon vous parvint assourdie~:\\
--- Et avec qui je vais parler moi~?\\

Devant l'heure avancée, le garde prit congé, s'excusant de mettre brutalement fin à la discussion.\\
--- Je vais devoir retourner ma ronde, ce fut un plaisir\\
--- Plaisir partagé~! répondit Narcisse\\
--- Au revoir~! le salua Pyracmon depuis sa geôle.\\
Le garde vérifia son épée, remit son casque et sortit dans la rue. Peu de temps après, les quatre soldats sérésiens se levèrent également, l'un d'eux se tourna vers nos trois compagnons~:\\
--- Nous allons devoir fermer, il faut que vous sortiez.\\
--- Soit. obtempéra Morgan
--- Et moi~? Je ne sors pas avec vous~? demanda tristement Pyracmon, toujours dans sa cellule\\
--- Hélas non, mais continuez à être attentif, c'est comme cela que vos facultés se développeront. le conseilla Narcisse.\\

Narcisse eut à peine le temps de finir sa phrase, que la table s'ébranla, faisant dangereusement tanguer une carafe de vin. Par chance les gardes étaient déjà dehors et tout cela leur échappa. Morgan souffla à Narcisse~:\\
--- Je crois que Pyracmon ne devrait pas trop s'appuyer sur les murs…\\
--- Oui, c'est un brave garçon, mais il est un peu maladroit.\\
Narcisse et Morgan quittèrent la caserne. Sans demander leur reste, les gardes s'éloignèrent à grands pas, sans doute vers une auberge voisine…\\
--- C'est maintenant qu'il faut que Pyracmon décide de s'évader… Croyez-vous, mon ami, que j'ai été trop subtil avec lui~?\\
--- Mais non regardez~: À L'AIIDDEE~! hurla Morgan.\\
L'effet ne se fit pas attendre~: les quatres gardes sérésiens ne tardèrent pas à revenir en courant tandis que Pyracmon traversait le mur de sa prison, un bout de pain à la main. Après avoir consciencieusement rangé son quignon, Pyracmon dégaina sa hache à deux mains et déclara~:\\
--- Je m'ennuyais un peu\\
Derrière lui, ce qui restait du mur s'écroula, emportant une partie du toit qui s'effondra sur un garde qui mourrut sur le coup. Tentant de faire une diversion, Narcisse s'approcha d'un des gardes en courant, se prit les pieds dans sa robe et s'écroula sur le sol. Les gardes stupéfaits tant par la chute inattendue de Narcisse que la fuite - peu discrète - de Pyracmon restèrent un instant immobile. Moment d'immobilité suffisant pour Morgan qui en profita pour décocher une flèche qui atteignit le second garde à la tête. Sans attendre de voir si le garde était mort ou non, Pyracmon décida de broyer le cadavre en hurlant un~\og PICOUGNAAAC \fg~terrifiant. Remis sur des pieds, Narcisse ne perdit par de temps et après quelques incantations dans une langue aussi oubliée que gutturale, des éclairs jaillirent de ses mains. Le troisième garde, foudroyé, s'écroula au sol. Devant le bain de sang, le dernier garde, épée au clair hésita, mais déjà une flèche lui transperça la cuisse tandis Pyracmon s'approchait, sa hache à deux mains levée bien haut. Quelques secondes et une explosion de sang et de chair plus tard, s'en était fini de la maréchaussée de Soll.\\

Morgan, jusque là accroupi se releva, poussant un cadavre du pied. Narcisse, incomodé par la vue de tant de sang proposa de s'éloigner un peu. Pyracmon accueillit l'idée avec joie après avoir piétiné un cadavre et nettoyé sa hache.

Nos trois héros déambulèrent un peu dans les rues à la recherche d'une sombre venelle pour manger un peu et se reposer.\\
--- Peut-êt'e pourrions-nous nous rapprocher de l'axche princhipal de la ville~? proposa Narcisse, la bouche pleine\\
La nuit était tombée depuis peu lorsque la petite, mais efficace, équipée arriva à la hauteur du convoi sérésien. Au du moins ce qu'il en restait. À la lueur fugace des torches, on devine le massacre en cours, les hommes de Frolorn dominent sans conteste la situation. Le sang gicle et une forte odeur de mort traîne dans l'air. Le restant de la maréchaussée, assiste, impuissante à l'échec de sa mission.\\
--- Ça sent le combat. nota Pyracmon avec gourmandise\\
--- Oh oui~! renchérit Morgan en dégainant sa dague ciselée.\\
Narcisse se fit spectateur et encouragea ses deux amis.\\

Pyracmon s'avança vers un garde sérésien, paralysé par la peur, celui-ci trébucha en arrière, escivant ainsi le coup de hache qui lui était destiné. Au sol, le garde tenta de percer le cuir de Pyracmon, mais échouant à sa tâche il finit découpé en deux dans le sens de la hauteur. Pyracmon, fier de son coup hurla~\og PYCOUGNAAAC~\fg. Et attaque derechef un autre garde. De son côté Morgan avait expédié un premier soldat au sol, lui plantant sa dague jusqu'à la garde sous la mâchoire. Narcisse qui s'aperçut que personne n'écoutait ses dissertations philosophiques, décida qu'il y avait un temps pour tout et que le moment était venu de taper sur du sérésien. Il s'avança donc et transperça un soldat de sa dague sans rencontrer de grande difficulté. Fouillant le cadavre il découvrit que le soldat était pauvre et en conclut qu'il ne méritait pas de vivre. Surprenant son geste, Morgan le héla.\\
--- C'est quand même très mal payé de tuer des pauvres…\\
--- À qui le dites-vous~! répondit Narcisse.\\
--- TAPERR~! hurla Pyracmon, moins philosophique mais plus efficace, en réduisant le volume d'un soldat d'un facteur au moins deux.\\






 




























\end{document}