\documentclass[french, a4paper, 12pt]{article}

\newcommand{\docname}{Empire d'Omar}

% Layout
\usepackage{fancyhdr, fancyvrb}
\usepackage{lastpage}
\usepackage[left=2.0cm, right=2.0cm, top = 2.0cm, bottom = 2.0cm]{geometry}
\usepackage{multicol}
\usepackage{graphicx}
\usepackage{hyperref}
\hypersetup{hyperindex=true, colorlinks, linkcolor=blue, urlcolor=blue, citecolor=blue, breaklinks=true}
\everymath{\displaystyle}


% Language
\usepackage[utf8]{inputenc}
\usepackage[T1]{fontenc}
\usepackage{babel}


% Pagestyle
\pagestyle{fancy}

\renewcommand{\headrulewidth}{0.4pt}
\renewcommand{\footrulewidth}{0.4pt}

\title{\sc \docname}
\date{}

\lhead{}
\rhead{\docname}
\cfoot{\thepage\ $|$ \pageref{LastPage}}
\rfoot{Odyssée}

\begin{document} \maketitle \vspace{3pt} \hrule \vspace{3pt}


\section{Chiffres}

\begin{verbatim}
Population ...........: 5.9 millions
Superficie ...........: 952 000 kilomètres carrés
Régime politique .....: Empereur assisté de Ministres
Capitale .............: Marari
Adjectif .............: omarien, omarienne
Culture ..............: Eldar, Romian, Northumbic, Kentian, Aj'Snaga
Religions pratiquées .:
\end{verbatim}

\section{Culture}

\subsection{Généralités}

La culture Eldar est très influente par son architecture. Néanmoins, plusieurs cultures cohabitent, notamment les cultures Romian et Kentian. Si la culture Eldar n'est pas très tolérante, la culture Romian est bien plus cosmoplite et offre un endroit plus paisible aux minorités persécutées.

Néanmoins, les milices omariennes traquent les Elfes, même dans ces zones, notamment grâce au 6ème régiment, basé à Victum.

\section{Société}

\subsection{Généralités}

La population est essentiellement constituée de bâtisseurs, qui comptent parmi les meilleurs au monde, et d'artisants. Le peuple omarien, dans les grandes lignes, suit les principes de la culture Eldar, une culture puissante et influente qui a conquis une bonne partie du globe. Les connaissances poussées des omariens en architecture ont sans doute joué un rôle important dans le rayonnement culturel Eldar.

\subsection{Les Bâtisseurs Eldar}

Considéré comme l'élite de la nation, les bâtisseurs forment un corps très bien formé dont l'enseignement repose sur l'équilibre des forces et des formes. Les bâtisseurs omariens sont souvent appelés à travailler à l'étranger, où de riches propriétaires sinon des États, font appel au savoir-faire omarien pour la réalisation de bâtiments.


\subsection{Transport fluvial}
	
Par l'évolution de leur culture, le peuple omarien n'est pas un peuple de navigateurs de grande mer. Plus proche du transport fluvial et des petits bâteaux légers à fond plats, l'économie omarienne souffre du manque d'ouverture maritime du pays. Néanmoins, les rivières et fleuves omariens sont bien exploités, ce qui permet un bon transit interne.

\subsection{Tensions}

La faiblesse de l'agriculture omarienne tend à créer des tensions entre le peuple du sud-ouest et l'État, ce dernier étant accusé de ne pas prendre les mesures nécessaires pour subvenir aux besoins de sa population. Néanmoins, aucune révolte n'a encore éclaté, même si le ton monte.

La population est également constituée d'une forte communauté elfique qui est traquée. Cette dernière est surtout présente dans la province du Teotlan, une grande île au nord du pays, sous influence Romian, où la culture leur est plus favorable. Cette persécution est également pour quelque chose dans le durcissement des rapport entre l'Empire d'Omar et ses voisins, qui sont plus tolérants vis à vis des Elfes.

\section{Politique}

\subsection{Généralités et tensions}

Le régime omarien est assez radical, les Elfes sont mal vus et pourchassés. De plus le gouvernement est loin du peuple, ce qui fait monter la tension entre l'État et ses sujets. Les Elfes sont suffisemment puissants pour monter le peuple contre le gouvernement, mais ils sont trop isolés pour cela, repliés dans les taïgas, ils fuient les persécutions.

\subsection{Système politique}

L'Empire d'Omar est dirigé par l'Impératrice Myranlean qui gouverne d'une main de fer avec un conseil consitué de quelques Ministres triés sur le volet.

\section{Économie}

\subsection{Généralités}

L'économie du pays est assez fragile malgrè sa taille. Si les bâtisseurs omariens, ne trouvent plus d'engagements, l'empire va vite chuter. L'artisanat, présent en faible quantité n'exporte qu'une très petite partie de sa production, et l'agriculture traditionnelle pratiquée au sud et à l'ouest suffit à peine à nourrir la population.

\subsection{Mines et métallurgie}

L'Empire d'Omar a deux mines connues : une de fer à Hluilon et une de cuivre vers Shercast. Cependant, les métaux sont juste extraits, le pays n'ayant pas d'école de métallurgie, la totalité de la production est exportée.

\section{Diplomatie}

\subsection{Historique}

La culture Eldar a créé des liens étroit entre l'Empire d'Omar et celui de Skorgenia. De même, l'Empire a établi de très bonne relation avec l'Heptarchie d'Orle, le Royaume de Seresia et le Grand Duché de Boisia. Pour autant, l'Empire d'Omar n'est pas tombé dans l'extrêmisme de Skorgenia et a également su garder des relations courtoise avec plusieurs ennemis de l'Empire de Skorgenia comme le Royaume de D'elkisad ou le Grand Duché de Willow.

\subsection{Situation actuelle}

La Principauté du Stanland et la Théocratie de Torkia ont récemment conclu un pacte d'aide mutuelle. Suite à cette annonce, l'Empire d'Omar s'est retrouvé coupé du reste du monde par voie de terre. Les nouveaux alliés n'ont pas explicitement coupé la voie à Omar, mais des rapports de guetteurs et de postes avancés mentionnent des mouvements de troupes de plus en plus fréquents. Cependant la majorité de l'armée Torko-stanlandaise est encore loin de la frontière omarienne.


\end{document}