\documentclass[french, a4paper, 12pt]{article}

\newcommand{\docname}{Royaume de Seresia}
\newcommand{\authorname}{Loris Delafosse et Antoine Royer}
\newcommand{\dmy}{}
\newcommand{\years}{2020 -- 2021}
\newcommand{\org}{Odyssée}

\input{~/latex/article.tex}

\begin{document} \maketitle \vspace{3pt} \hrule \vspace{3pt}

\tableofcontents



\section{Chiffres}

\begin{verbatim}
Population ...........: 3.1 millions
Superficie ...........: 540 000 kilomètres carrés
Régime politique .....: Monarchie absolue de droit divin
Capitale .............: Vilese
Adjectif .............: seresien, seresienne
Culture ..............: Luarian
Religions pratiquées .: Témoins de Griverdeneres
\end{verbatim}

\section{Culture}


\subsection{Envie de conquête}

La culture luarienne est une branche dévirée du Trow, mais qui ne connaît pas la même gloire et le même rayonnement. Le peuple nourrit une forme de fierté pour cette culture, et il aimerait la répandre, au mépris du message initial de la culture luarienne qui est plutôt pacifique.

\subsection{État et religion}

La religion, avec laquelle le gouvernement est très lié, n'arrange pas les choses. En effet, les témoins de Griverdeneres est une religion qui idolâtre la figure de Griverdeneres, une vieille femme aveugle dont la générosité n'a d'égale que ses pouvoirs. Au nom de l'amour porté à ce symbole, le peuple est prêt à tout.

\section{Société}

\subsection{Importance de la culture et de la religion}

\subsubsection{Poids de la religion}

Tout le peuple est maintenu par le poids de la religion. La culture, a priori plutôt ouverte aux autres, et la religion dont la figure principale est caractérisé par sa générosité, ont été défiguré par le temps pour servir le peuple.

\subsubsection{Conséquences et tensions}

Le peuple a ainsi développé une forme de violence, alimentée par la religion et le besoin vital de répandre la culture luarienne sans en respecter les principes fondamentaux. Le gouvernement a déjà tenté de réfréner les pulsions de la population sérésienne, mais la tâche s'est révélée trop longue et complexe.

\subsection{Organisation et système éducatif}

Le gouvernement détient des Académies dans chaque ville importante (soit une trentaine en tout). Ces Académies proposent des enseignements variés et de qualités variables selon les lieux. L'Académie de Guning passe pour être la meilleure du pays avec un enseignement du combat naval qui est reconnu des pays limitrophes. Les autres matières sont classiques, on citera toutefois la présence d'architectes de marine.

\section{Politique}

\subsection{Généralités}

La monarchie de droit divin est encore un signe du poids de la religion sur le peuple. Le gouvernement œuvre à la prospérité du royaume et le peuple n'est pas malheureux.

\subsection{Tensions passées et présentes}

Vers l'an 150, lorsque la violence du peuple a commencer à déranger, le roi, qui s'appellait alors Manter, a d'abord tenté de mettre un terme à l'agitation des rues. Cependant, il avait alors bien conscience qu'une intervention trop brutale lui vaudrait les foudres de sa population, ce qui n'était pas souhaitable. Coincé dans une impasse, entre calmer les seresiens sans pour autant envoyer l'Armée Royale, Manter fit alors le choix de tourner la colère de la population pour servir les intérêts du Royaume gageant sur le fait qu'il n'y pas plus puissante armée qu'une troupe de fanatiques encolérés.

Encore aujourd'hui, le peuple conserve cette fougue et cette force.

\section{Économie}

\subsection{Puissance principale}

La source de revenue principale est la vente de bâteaux. Lourds gallions, frégates légère, sveltes shooners, toues cabanées, péniches… L'industrie portuaire est très développée et l'art du combat naval est maîtrisé par les seresiens qui tirent une certaines fierté de ce savoir.

\subsection{Reste du pays}

Cependant les littoraux représentent une minorité de personnes et la plupart de la population active n'est pas employée dans la fabrication de bâteaus ou dans les activités dérivées (exploitation de bois, métallurgie…). La population intérieure est ainsi délaissée par la capitale, car jugée peu productive. Cette population constitue en fait le grenier à blé du pays.

\section{Diplomatie}

\subsection{L'armée}

L'armée seresienne a une activité moyenne, avec 26 000 soldats, archers et marins seulement (soit 0.84% de la population), Seresia ne fait pas partie des plus grandes puissances militaires du monde. Cependant, le Royaume de Seresia a la deuxième flotte la plus importante.

\subsection{Relations diplomatiques}

Grand ami de la Principauté du Stanland, de l'EMpire d'Omar et du Royaume de D'elkisad, suzerin de l'Heptarchie d'Orle et du Grand Duché de Boisia, le Royaume de Seresia est bien entouré. Cependant, des rivalités dûes à la religion et à la culture existent, ce qui froisse les relation avec le Grand Duché de Willow. Ce dernier sort d'une guerre de religion durant laquelle le Grand Duché a une partie de son territoire sous contrôle D'elkisadais. Depuis Willow ne cesse de renforcer son armée et si une guerre venait à éclater, Seresia aurait beaucoup de mal à suivre.

\subsection{Relation complexes avec le Grand Duché de Boisia}

L'économie du Royaume repose sur l'industrie navale laquelle nécessite beaucoup de bois. Profitant de la détresse politique et économique du Grand Duché de Boisia (territoire riche en ressources forestières, mais peu exploité), Seresia a profiter de la révolution boisaise pour prendre le contrôle de la peninsule. En faisant couler l'économie boisaise, Seresia affaiblit tout le pays et pourra annexer le pays sans grande difficultés afin de pouvoir exploiter pleinement le bois mais aussi les quelques 3800 km de façade maritime du Grand Duché.



\end{document}