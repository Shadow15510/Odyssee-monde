\documentclass[french, a4paper, 12pt]{article}

\newcommand{\docname}{Empire d'Otlin}

% Layout
\usepackage{fancyhdr, fancyvrb}
\usepackage{lastpage}
\usepackage[left=2.0cm, right=2.0cm, top = 2.0cm, bottom = 2.0cm]{geometry}
\usepackage{multicol}
\usepackage{graphicx}
\usepackage{hyperref}
\hypersetup{hyperindex=true, colorlinks, linkcolor=blue, urlcolor=blue, citecolor=blue, breaklinks=true}
\everymath{\displaystyle}


% Language
\usepackage[utf8]{inputenc}
\usepackage[T1]{fontenc}
\usepackage{babel}


% Pagestyle
\pagestyle{fancy}

\renewcommand{\headrulewidth}{0.4pt}
\renewcommand{\footrulewidth}{0.4pt}

\title{\sc \docname}
\date{}

\lhead{}
\rhead{\docname}
\cfoot{\thepage\ $|$ \pageref{LastPage}}
\rfoot{Odyssée}

\begin{document} \maketitle \vspace{3pt} \hrule \vspace{3pt}

\section{Chiffres}

\begin{verbatim}
Population ...........: 7.8 millions
Superficie ...........: 1.2 millions de kilomètres carrés
Régime politique .....: Monarchie héréditaire
Capitale .............: Elkegkan
Adjectif .............: otlinois, otlinoise
Culture ..............: Yotunn, Romian
Religions pratiquées .:
\end{verbatim}

\section{Culture}

\subsection{Généralités}

La culture Yotunn prédomine, mais l'Île, au nord-est du pays au large de Elkegkan, est plus influencée par la culture Romian, un dérivé direct de la première.

\section{Société}

\subsection{Population et territoire}

Avec 7.8 millions d'otlinois et 1.2 millions de kilomètres carrés, c'est le pays le plus grand et le plus peuplé. Le peuple otlinois est réputé pour son pacifisme et son sens aigu de l'hospitalité.

\subsection{Tension internes}

Par le passé il y a eu des tensions entre le peuple et le gouvernement à cause du régime autocratique. Ces tensions ce sont surtout manifestées dans les zones urbanisées au nord, mais aussi au sud où la population, proche de la frontière s'est sentie menacée par l'aggressivité croissante de l'Empire de Skorgenia. L'État a récemment mobilisé des unités pour protéger des populations qui se sentait délaissées devant la menace skorgaise, et les liens entre peuple et gouvernement se sont resserés. Pour autant le régime politique reste loin d'être une démocratie.

\section{Politique}

\subsection{Historique}

Le régime fut un temps dur et autoritaire, mais le territoire est si grand qu'appliquer la dictature sur tout le territoire s'est révélé une impasse. Une milice avait été créée, mais l'hospitalité paysanne, rude et chaleureuse, a eu raison des miliciens. Le gouvernement s'est retrouvé démuni et s'est trouvé matériellement contraint d'assouplir le régime en place.

Ce choix relève plus une question de relations internationnales, en effet les populations rurales éloignées du centre de l'État (i.e. : le sud du pays) n'ont pas vu beaucoup de changements. Le niveau de vie est assez faible en ville, les campagnes sont finalement plus agréables à vivre : nourriture, hygiène sommaire mais néanmoins meilleure loin de la puanteurs des villes.

\subsection{Système politique}

La gouvernance est assurée par un monarque de droit héréditaire.

Les questions relatives à la succession sont réglées dans un document précieusement conservé à la Bibliothèque Impériale d'Elkegkan : la "Règle de Tagdar" ou "Tradition de Tagdar", antique poème de 6500 vers écrits en Haut-Yotunn, que la légende attribue au mythique premier souverain du pays, et qui fait office de loi fondamentale de l'Empire. L'accession au trône est un véritable marathon protocolaire pour le Prince héritier (le fils ainé de l'Empereur, sauf dans les situations dites "d'indignité" dont la Règle de Tagdar dresse la liste exhaustive) : le Prince doit en effet se soumettre à pas moins de 37 cérémonies au cours des deux jours et deux nuits que dure la succession, parmi lesquelles un examen médical qui consiste principalement à constater que le Prince "est doté de dents dont la blancheur nacrée est apte à conduire l'Empire tel un phare sur l'horizon immense" selon la formule consacrée prononcée par deux docteurs de l'Université vêtus pour l'occasion de robes à crinoline vertes frappées du sceau de la Faculté de Médecine. Dans le même ordre d'idées, un "examen d'aptitude intellectuelle" est également mené sous la conduite du Vénérable Recteur de l'Université, au cours duquel 80 questions rituelles (également listées dans la Règle de Tagdar) sont posées au Prince. A l'issue de ces deux examens, le Recteur s'adresse à la foule en proclamant "Eg meer silep !", formule rituelle dont le sens est encore aujourd'hui l'objet de débats parmi les savants otlinois, certains y comprenant "Il est sain" et d'autres "Puisse-t-il gouverner sainement". La cérémonie du couronnement et les diverses cérémonies d'hommage visant à garantir le respect de la sacrosainte autorité impériale par tous ses sujets sont naturellement les moments les plus marquants de la succession.

Le caractère divin de l'Empereur (qui est en effet vénéré comme un dieu du Panthéon de Rangorhe) l'emporte généralement sur la tradition quand il s'agit de la gouvernance de l'Empire, et il n'est pas rare que l'organigramme du gouvernement central change radicalement d'un Empereur à l'autre : existence, importance et autonomie d'un Grand Conseil, d'un Tribunal Impérial, de ministères… sont susceptibles de considérables variations. En revanche, le gouvernement des provinces est laissé aux *préfets impériaux*, qui tiennent leur pouvoir directement de l'Empereur et peuvent être révoqués à tout moment. La tradition veut toutefois que les préfets impériaux de chaque province soient choisis dans une famille déterminée, et la charge est ainsi, souvent, *de facto* héréditaire

\section{Économie}

\subsection{Généralités}

L'économie est basée sur l'agriculture et l'exploitation du bois. Le pays n'a aucune mine connue et doit importer les métaux qu'elle souhaite utiliser, elle n'est donc pas autonome sur le plan militaire. De même l'orfévrerie ne fait pas partie des spécialités de l'empire otlinois.

Néanmoins les marchands Otlinois sont connus du monde entier, les cartographes sont réputés, de manière plus minoritaire, les marins otlinais sont bons, même si ils ne font pas partis des meilleurs.

\subsection{Système universitaire}

L'Université Impériale d'Elkegkan créée il y a quelques dizaines d'années est une école principalement marchande qui forme cartographes et navigateurs. Récemment un nouveau département a été fondé dans le but de mieux gérer la consommation de bois. Les ressources commencent à se raréfier et l'État s'inquiète de l'avenir de l'économie otlinoise.

Bien sûr, l'Université regroupe des départements de théologie, de sciences ou encore de philosophie qui sont reconnus. La Grande Bibliothèque Impériale d'Elkegkan passe pour être la plus fournie au monde.

Cependant, si l'Université a été, fût un temps, à la pointe de la recherche, cela fait longtemps que les autres pays ont lancés leurs propres recherches et aujourd'hui, le pays garde le charme de l'ancien et du traditionnel tout en supportant les conséquences.

\section{Diplomatie}

\subsection{Généralités}

L'Empire d'Otlin est le pays le moins actif sur le plan militaire. Sa position géographique, un peu excentrée en fait un allié de peu d'intérêt.

\subsection{Relations internationales et alliances}

Sans grand allié, l'Empire d'Otlin s'est construit replié sur lui-même. Il a de vagues relation amicales avec le Royaume de D'elkisad, plus impliqué au niveau internationnal. À l'inverse, les délires expansionniste de l'Empire de Skorgenia induisent des rapports légèrements tendus entre les deux pays sans pour autant dégénérer en guerre ouverte.

\subsection{Tensions actuelles}

La recherche otlinoise est maintenant dépassée par les derniers progrès modernes, si une guerre venait à se déclencher contre l'empire skorgais, l'Empire d'Otlin souffrirait beaucoup de ce retard et de son isolement.

\subsection{Guerre contre de l'Empire de Skorgénia}

\subsubsection{Contexte}
L'Empire de Skorgénia a récemment entreprit d'envahir le sud de d'Otlin pour étendre son influence. Le combat se déroulant majoritairement dans des petits villages montagnars, Otlin a été long à réagir et n'a envoyé que peu d'hommes, mobilisant un unique régiment de 1600 hommes alors en garnison à Gigezruc, situé à 300 kilomètres au nord de la zone de conflits (le plus proche).

Pendant ce temps le Royaume a massé près de 2000 hommes réparti en deux régiments distincts et chacun attaquant un endroit différent. La zone occupée ne cesse de s'étendre (aux dernières nouvelles la zone sous contrôle avoisinnait les 38~000 kiloèmtres carrés) et le plus proche régiment Otlinois se trouve à plus 1500 kilomètres.

\subsubsection{Bataille de la rivière de Kinzarkud}
Le détachement skorkais fort d'un millier d'hommes ne pensait pas se faire attaquer par un régiment plus grand. L'attaque de l'Empire d'Otlin a été rapide et efficace. L'armée skorgaise a essuyé une cuisante défaite, mais le régiment otlinois a été réduit aussi. Néanmoins, cette bataille restera une humiliante défaite pour l'armée skorgaise.




\end{document}